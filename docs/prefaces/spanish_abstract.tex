\documentclass[../main.tex]{memoir}
\begin{document}
\thispagestyle{empty}

\begin{center}
  {\large\bfseries \ProjectTitle}\\
\end{center}

\begin{center}
  \AuthorName\\
  \vspace{0.7cm}
  \noindent{\textbf{Palabras clave}:}\\
  Aprendizaje profundo, análisis de multitudes, detección de
  anomalías, videovigilancia\\
  \vspace{0.7cm}
  \noindent{\textbf{Resumen}}\\

\end{center}

En las últimas décadas se ha producido un crecimiento poblacional sin
precedentes en todas las partes del mundo, y las tasas de criminalidad
y terrorismo se han disparado en muchos territorios. Esto ha provocado
que la videovigilancia se convierta en una herramienta prioritaria a
nivel mundial. El número de cámaras de seguridad instaladas tanto en
ámbito público como privado ha crecido significativamente, y con ello
la dificultad de gestionar la información recogida por las mismas de
forma manual. Aparece, por tanto, la necesidad de automatizar este
proceso, utilizando para ello modelos inteligentes capaces de extraer
información de las secuencias de vídeo recogidas por las cámaras.\\

Los avances en aprendizaje profundo de los últimos años han permitido
que los resultados obtenidos sobre esta área de investigación mejoren
considerablemente. Los modelos actuales son capaces de extraer
información más compleja, y trabajar en entornos de mayor dificultad,
especialmente en cuanto a densidad de individuos se refiere. A pesar
de ello, el estudio de esta tarea es relativamente reciente, por lo
que no está correctamente estructurada, y resulta difícil comparar los
trabajos propuestos.\\

Dado el contexto anterior, este trabajo trata de resolver varias
tareas relacionadas con el tratamiento automático de fuentes de
videovigilancia, en particular con el análisis de comportamientos de
multitudes. Por un lado, se realiza un estudio teórico de la temática,
con una propuesta taxonómica que permite agrupar los distintos
trabajos siguiendo una secuencia de tareas. Esta organización sitúa
las distintas subtareas consideradas dentro de la temática en
distintos pasos de la secuencia, de forma que los resultados de los
pasos posteriores se ven fuertemente influenciados por los obtenidos
en los pasos previos.\\

Además de la propuesta taxonómica, en el estudio teórico se hace una
revisión exhaustiva de la literatura que utiliza modelos de
aprendizaje profundo para resolver el problema de la detección de
anomalías en multitudes. Para esta subtarea, se analizan los
principales conjuntos de datos disponibles públicamente, y se estudian
los trabajos del estado del arte, agrupando los mismos por el tipo
concreto de anomalía que tratan de identificar.\\

En el apartado práctico del trabajo, se estudia el uso de
características espacio-temporales para la detección de acciones
anómalas en vídeo. Para llevar a cabo dicho estudio, se establece como
punto de partida un modelo de detección de anomalías basado en un
extractor de características convolucional en tres
dimensiones. Nuestra propuesta, en lugar de utilizar un extractor
exclusivamente convolucional, aprovecha la potencia de las redes
neuronales convolucionales 2D para el análisis de los fotogramas por
separado, extrayendo información espacial, y la capacidad de las redes
neuronales recurrentes para extraer información temporal de la
secuencia de características de los fotogramas consecutivos. Dicho
extractor de características es más complejo que la propuesta
original, y conserva mejor la estructura temporal del vídeo, lo cual
permite la extracción de información de mayor calidad. El código
desarrollado para el experimentación se encuentra disponible en el
repositorio \url{https://github.com/fluque1995/tfm-anomaly-detection}.\\

Los resultados obtenidos del estudio sugieren que nuestro modelo tiene
mejor capacidad de clasificación que el modelo original, incluso a
pesar de estar entrenado en un conjunto de datos de tamaño 1000 veces
menor que el extractor de características de partida. Este hecho
nos permite concluir que nuestra propuesta es de mayor calidad que el
modelo de partida, validando nuestra hipótesis inicial.

\newpage
\end{document}

%%% Local Variables:
%%% mode: latex
%%% TeX-master: "../main"
%%% End:
